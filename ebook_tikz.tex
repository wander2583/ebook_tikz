\documentclass[12pt,oneside]{book}

% --- Pacotes essenciais ---
\usepackage[utf8]{inputenc}
\usepackage[T1]{fontenc}
\usepackage{lmodern}
\usepackage[english,brazil]{babel}
\usepackage{geometry}
\usepackage{graphicx}
\usepackage{tikz}
\usepackage{hyperref}
\usepackage{fancyhdr}
\usepackage{titlesec}
\usepackage{setspace}
\usepackage{pdfpages}

% --- Geometria da página ---
\geometry{
    a4paper,
    top=2.5cm,
    bottom=2.5cm,
    left=3cm,
    right=2.5cm
}

% --- Estilo da página ---
\pagestyle{fancy}
\fancyhf{}
\fancyfoot[C]{\thepage}

% --- Espaçamento entre linhas ---
\onehalfspacing

% --- Hiperlink ---
\hypersetup{
    colorlinks=true,
    linkcolor=blue,
    urlcolor=blue,
    pdftitle={Aprendizado sobre o TikZ},
    pdfauthor={Wanderlei Rodrigo Pereira},
    pdfsubject={TikZ para LaTeX},
    pdfkeywords={TikZ, LaTeX, Gráficos, Ebook, Técnicas, Diagramas}
}

% --- Capa personalizada com imagem ---
\usepackage{eso-pic}
\newcommand\BackgroundPic{
    \put(0,0){
        \parbox[b][\paperheight]{\paperwidth}{
            \vfill
            \centering
            \includegraphics[width=\paperwidth,height=\paperheight,keepaspectratio]{capa.jpg}
            \vfill
        }
    }
}

\begin{document}

% --- Página de Capa ---
\begin{titlepage}
\AddToShipoutPicture*{\BackgroundPic}
\vspace*{15cm}
\begin{center}
    {\Huge\bfseries\textcolor{white}{Aprendizado sobre o TikZ}}\\[1cm]
    {\Large\textcolor{white}{Wanderlei Rodrigo Pereira}}
\end{center}
\end{titlepage}

% --- Página em branco após capa ---
\cleardoublepage
\thispagestyle{empty}
\null
\cleardoublepage

% --- Sumário ---
\tableofcontents
\cleardoublepage

% --- Capítulos de exemplo ---
\chapter{Introdução}
O TikZ é uma ferramenta poderosa para criação de gráficos vetoriais no LaTeX...

\chapter{Conceitos Básicos}
Explicação dos comandos fundamentais...

\chapter{Exemplos Práticos}
\begin{tikzpicture}
    \draw[->] (0,0) -- (2,0);
    \draw[->] (0,0) -- (0,2);
    \node at (1,1) {Exemplo};
\end{tikzpicture}

% --- Final ---
\end{document}
